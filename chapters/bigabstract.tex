The school-to-work transition (SWT) refers to the phase during which an individual exits the education system and and begins employment. This transition is considered a key developmental step in the individual's transition to adulthood. In many low-income countries (LICs), and particularly in Sub-Saharan Africa (SSA), a stagnating formal sector and ever-growing youth cohorts impede the SWT and hinder young individuals' access to gainful employment. Delayed SWTs imply long periods of inactivity, which are known to create lasting, negative wage and employment effects. Young women are known to be particularly disadvantaged in their SWT, facing constricting social norms surrounding occupation choice and family formation.

The SWT path leads through the informal sector for the vast majority of African youth. This dissertation focuses on characterizing the SWT in highly informal economies, by means of a cross-country comparison of youth labor market strength across lower-middle income countries (LMICs) and LICs (Chapter 1), the measurement and mapping of SWTs in an urban setting (Chapter 2), and the evaluation of the costs and benefits of a national apprenticeship scheme embedded in the informal sector (Chapter 3).

In Chapter 1 (co-authored), the focus is on constructing and analyzing the composite Youth Labor Market Index for Low-Income Countries (YLILI). By utilizing a comprehensive set of indicators across three dimensions -- transition, working conditions, and education -- the index provides a nuanced evaluation of youth-specific labor market strength. Examining data from a range of low and lower-middle income countries, the analysis reveals substantial variation in youth labor market conditions and underscores the importance of education quality as a key driver of labor market outcomes.

Chapter 2 (single-authored) shifts the focus to the transitions from education to the workforce in the context of an urban labor market dominated by the informal sector. Drawing on longitudinal data from the economic center of Bénin, the city of Cotonou, this study examines the trajectories of young individuals through various activity states. Employing diverse methodologies, this chapter highlights the late school-leaving age of urban youth, the low probability of exiting the SWT observed for young women, and the low permeation between activity states and consequent importance of sorting in the early stages of SWT path.

Chapter 3 (co-authored) presents a cost-benefit analysis of the \textit{Certificat de Qualification Professionnelle} (CQP), a national apprenticeship program in Benin. This training scheme introduces a novel approach by combining classroom education with hands-on training, embedded in traditional apprenticeships in the informal sector. The study evaluates the impact of the CQP program on apprentices' skill development and labor market outcomes. The analysis also assesses the program's costs and benefits for both participating apprentices and the firms providing the training.

The findings suggest that addressing the challenge of prolonged youth labor market transitions in highly informal economies necessitates three interconnected strategies. First, reducing youth cohort sizes are a pressing necessity, as high fertility rates delay demographic dividends, reduce female labor force participation, and intensify competition for limited employment opportunities. Second, the findings highlight the need for enhanced data quality and availability in order to better understand the nuances of school-to-work transitions, in particular by leveraging longitudinal data. Lastly, policies aimed at integrating young women into labor markets are found to be pivotal for fostering economic independence and inclusive growth.


