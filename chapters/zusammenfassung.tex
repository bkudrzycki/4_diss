Der Übergang von der Schule ins Berufsleben (SWT) bezeichnet die Phase, in der eine Person das Bildungssystem verlässt und eine Beschäftigung aufnimmt. Dieser Übergang wird als ein wichtiger Entwicklungsschritt beim Übergang des Einzelnen ins Erwachsenenalter angesehen. In vielen einkommensschwachen Ländern (LICs) und insbesondere in Afrika südlich der Sahara (SSA) behindern ein stagnierender formeller Sektor und ständig wachsende Jugendkohorten den SWT und erschweren jungen Menschen den Zugang zur Erwerbstätigkeit. Ein verzögerter SWT bedeutet lange Zeiten der Nichterwerbstätigkeit, die sich bekanntermaßen dauerhaft negativ auf die Löhne und die Beschäftigung auswirken. Es ist bekannt, dass junge Frauen in ihrem SWT besonders benachteiligt sind, da sie mit einschränkenden sozialen Normen in Bezug auf Berufswahl und Familiengründung konfrontiert sind.

Der Weg der SWT führt für die große Mehrheit der afrikanischen Jugendlichen über den informellen Sektor. Diese Dissertation konzentriert sich auf die Charakterisierung des SWT in hochgradig informellen Volkswirtschaften durch einen länderübergreifenden Vergleich der Arbeitsmarktstärke von Jugendlichen in Ländern mit niedrigem bis mittlerem Einkommen (LMICs) und LICs (Kapitel 1), die Messung und Kartierung von SWTs in einem städtischen Umfeld (Kapitel 2) und die Bewertung der Kosten und des Nutzens eines in den informellen Sektor eingebetteten nationalen Ausbildungsprogramms (Kapitel 3).

In Kapitel 1 (als Co-Autor erfasst) liegt der Schwerpunkt auf der Erstellung und Analyse des zusammengesetzten Jugendarbeitsmarktindex für Länder mit niedrigem Einkommen (YLILI). Durch die Verwendung eines umfassenden Satzes von Indikatoren in drei Dimensionen - Übergang, Arbeitsbedingungen und Bildung - bietet der Index eine nuancierte Bewertung der jugendspezifischen Arbeitsmarktstärke. Die Analyse von Daten aus einer Reihe von Ländern mit niedrigem und niedrigem bis mittlerem Einkommen zeigt erhebliche Unterschiede in den Arbeitsmarktbedingungen für Jugendliche auf und unterstreicht die Bedeutung der Bildungsqualität als Schlüsselfaktor für die Arbeitsmarktergebnisse.

Kapitel 2 (Einzelautor) verlagert den Schwerpunkt auf die Übergänge von der Ausbildung in den Beruf im Kontext eines städtischen Arbeitsmarktes, der vom informellen Sektor dominiert wird. Die Studie stützt sich auf Längsschnittdaten aus dem Wirtschaftszentrum von Bénin, der Stadt Cotonou, und untersucht den Werdegang junger Menschen in verschiedenen Aktivitätsstadien. Durch den Einsatz verschiedener Methoden werden in diesem Kapitel das späte Schulabschlussalter der städtischen Jugendlichen, die geringe Wahrscheinlichkeit des Ausstiegs aus dem SWT für junge Frauen und die geringe Durchlässigkeit zwischen den Tätigkeitsbereichen und die daraus folgende Bedeutung der Sortierung in den frühen Phasen des SWT-Pfads hervorgehoben.

Kapitel 3 (Co-Autor) präsentiert eine Kosten-Nutzen-Analyse des \textit{Certificat de Qualification Professionnelle} (CQP), eines nationalen Ausbildungsprogramms in Benin. Mit diesem Ausbildungsprogramm wird ein neuartiger Ansatz verfolgt, bei dem die Ausbildung im Klassenzimmer mit einer praktischen Ausbildung kombiniert wird, die in die traditionelle Lehrlingsausbildung im informellen Sektor eingebettet ist. Die Studie bewertet die Auswirkungen des CQP-Programms auf die Qualifikationsentwicklung der Auszubildenden und die Ergebnisse auf dem Arbeitsmarkt. Die Analyse bewertet auch die Kosten und den Nutzen des Programms sowohl für die teilnehmenden Auszubildenden als auch für die ausbildenden Unternehmen.

Die Ergebnisse deuten darauf hin, dass zur Bewältigung des Problems der langwierigen Übergänge von Jugendlichen in den Arbeitsmarkt in stark informellen Volkswirtschaften drei miteinander verbundene Strategien erforderlich sind. Erstens ist es dringend erforderlich, die Größe der Jugendkohorten zu verringern, da hohe Geburtenraten die demografische Dividende verzögern, die Erwerbsbeteiligung von Frauen verringern und den Wettbewerb um begrenzte Beschäftigungsmöglichkeiten verschärfen. Zweitens unterstreichen die Ergebnisse die Notwendigkeit einer verbesserten Datenqualität und -verfügbarkeit, um die Nuancen des Übergangs von der Schule ins Berufsleben besser zu verstehen, insbesondere durch die Nutzung von Längsschnittdaten. Und schließlich wird festgestellt, dass politische Maßnahmen zur Integration junger Frauen in den Arbeitsmarkt für die Förderung wirtschaftlicher Unabhängigkeit und integrativen Wachstums von zentraler Bedeutung sind.